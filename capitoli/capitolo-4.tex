% !TEX encoding = UTF-8
% !TEX TS-program = pdflatex
% !TEX root = ../tesi.tex

%**************************************************************
\chapter{Indici di prestazione dello stage}
\label{cap:performance-stage}
%**************************************************************

\section{Come abbiamo organizzato l'attività di stage}
%**************************************************************
\subsection{Obiettivi dello stage}
\subsubsection{La scelta degli obiettivi}
%**************************************************************
\subsubsection{Aspettativa vs realtà}
Mostra come la sicurezza di riuscire a soddisfare tutti requisiti, si sia tramutata poi in cura rispetto a quelli che erano gli obbligatori per portare al raggiungimento della finalità dello stage ovvero quello di incremento qualitativo del lavoro svolto
%**************************************************************
\subsubsection{Soddisfacimento degli obiettivi}
Espone il livello raggiunto rispetto agli obiettivi prefissati
%**************************************************************
\subsection{Cicli di sviluppo: Micro vs Macro}
Mostra come confrontandoci all'interno del team, siamo passati da un piano di progetto "a ciclo unico" a diversi cicli di sviluppo via via che si formalizzavano i requisiti e veniva verificata la fattibilità di ogni singola parte
%**************************************************************
\subsubsection{Andamento}
La sezione espone cosa ho percepito nei cambiamenti di ritmo fra i lenti momenti di confronto e le veloci realizzazioni di POC e codice
\subsubsection{Vantaggi}
Risponde a: Come ha influito la riduzione della dimensione dei cicli di sviluppo sulla qualità tramite frequenti verifiche?
%**************************************************************
\section{Analisi di profitto}
\subsection{Conoscere un nuovo ambito}
Bilancio del mio primo aproccio al Process Mining (obiettivo personale)

%**************************************************************
\subsection{Teamworking informatico}
Sezione in cui parlo della prima esperienza di teamworking in ambito informatico (fuori dal contesto scolastico) e di cosa mi ha arricchito.

%**************************************************************
\subsection{Ascoltare e comunicare}
Mostra come in diverse occasioni ci sia stato modo di imparare ascoltando gli altri membri del team e come in altre occasioni ci siano stati dei momenti di formazione da parte mia su specifiche componenti tecnologiche da me preparate per lo svolgimento dello stage.

%**************************************************************
\subsection{Documentarsi e imparare "in azione"}
Spiego che la cosa più importante che ho imparato è l'importanza di documentarsi in maniera completa qualora possibile, per trarne vantaggio anche in situazioni future che si possono creare. Dove questo non sia possibile, bisogna ricercare le parti critiche, magari tramite la creazione di POC e, dove insorgono insicurezze, andare selettivamente a imparare quelle che sono le competenze da acquisire.

%**************************************************************
\subsection{Il mio contributo al team}
Sezione in cui espongo cosa delle mie conoscenze preliminari ho prestato al team e perché sono state apprezzate

%**************************************************************
\section{Conclusione del percorso}
\subsection{Come prepararsi al lavoro}
Indico quali sono per me le competenze che non vengono fornite nel corso degli studi e che vengono però naturalmente assorbite durante l'attività lavorativa.

%**************************************************************
\subsection{Un corretto way of thinking dello studente di Informatica Unipd}
La mia opinione sulla giusta strada per ottenere il massimo da questo corso di laurea.

%**************************************************************