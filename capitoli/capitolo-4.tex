% !TEX encoding = UTF-8
% !TEX TS-program = pdflatex
% !TEX root = ../tesi.tex

%**************************************************************
\chapter{Indici di prestazione dello stage}
\label{cap:performance-stage}
%**************************************************************

\section{Come si è svolta l'attività di stage}
%**************************************************************
\subsection{Andamento}
La sezione espone cosa ho percepito nei cambiamenti di ritmo fra i lenti momenti di confronto e le veloci realizzazioni di POC e codice
%**************************************************************
\subsection{Aspettativa vs realtà: il completamento degli obiettivi}
Mostra come la sicurezza di riuscire a soddisfare tutti requisiti, si sia tramutata poi in cura rispetto a quelli che erano gli obbligatori per portare al raggiungimento della finalità dello stage ovvero quello di incremento qualitativo del lavoro svolto
%**************************************************************
\subsection{Cicli di sviluppo: Micro vs Macro}
Mostra come confrontandoci all'interno del team, siamo passati da un piano di progetto "a ciclo unico" a diversi cicli di sviluppo via via che si formalizzavano i requisiti e veniva verificata la fattibilità di ogni singola parte

%**************************************************************
\section{Analisi di profitto}
%**************************************************************
\subsection{Teamworking informatico}
Sezione in cui parlo della prima esperienza di teamworking in ambito informatico (fuori dal contesto scolastico) e di cosa mi ha arrichito
%**************************************************************
\subsection{Il mio contributo al team}
Sezione in cui espongo cosa delle mie conoscenze preliminari ho prestato al team e perché sono state apprezzate
%**************************************************************
\subsection{Un corretto way of thinking dello studente di Informatica Unipd}
La mia opinione sulla giusta strada per guadagnare il massimo da questo corso di laurea.