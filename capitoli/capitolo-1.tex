% !TEX encoding = UTF-8
% !TEX TS-program = pdflatex
% !TEX root = ../tesi.tex

%**************************************************************
\chapter{Introduzione}
\label{cap:introduzione}
%**************************************************************
\section{La realtà Siav S.p.a.}

Presentazione di massima dell'azienda Siav contenente ambiti di sviluppo, reparti e divisioni

%**************************************************************
\section{Il reparto ricerca e sviluppo}

Presentazione del mio reparto, di cosa si occupa, che tipologie di attività effettua
%**************************************************************
\subsection{Il way of working del reparto}
Spiegazione delle tecnologie e processi in uso in reparto

%**************************************************************
\subsection{Formazione continua}
Spiego come frequenti sessioni di formazione ( nel particolare del periodo in cui ero impiegato in siav, principalmente sui servizi AWS) portano il reparto a saper riconoscere e discutere al meglio delle soluzioni per ogni problema che si pone di risolvere.

%**************************************************************
\section{Produzione tramite sperimentazione}
Presentazione della filosofia che muove le molteplici attività i stage che Siav effettua in collaborazione con le università e come essa porti valore aggiunto al prodotti stesso.
Presentazione del concept software "Bipod", di chi ne è il proponente e da cosa è composto.
Spiegazione del perché la natura di Bipod è sperimentale (ovvero perché argomento di diversi stage)
%**************************************************************
