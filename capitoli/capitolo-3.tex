% !TEX encoding = UTF-8
% !TEX TS-program = pdflatex
% !TEX root = ../tesi.tex

%**************************************************************
\chapter{Una piattaforma solida}
\label{cap:modalita-svolgimento}
%**************************************************************
\section{Metodo di sviluppo}
%**************************************************************
\subsection{L'uso di cicli di sviluppo}
Spiega come sono stati utilizzati i molteplici cicli di sviluppo

%**************************************************************
\subsection{Diagrammi e formalizzazione}
Spiega cosa è stato modellato dagli use case e dall'analisi dei requisiti e come è stato fatto

%**************************************************************
\subsection{POC}
Spiega che POC sono stati creati, quale era il loro scopo e come sono stati poi trasformati in prodotto.

%**************************************************************
\subsection{Confronto continuo}
Spiega come il confronto continuo su tutto il materiale prodotto abbia influenzato, sopratutto nelle prime settimane, l'andamento dello stage e la solidità dell'idea del prodotto finale che si consolidava.
%**************************************************************
\subsection{Documentazione ampia e prolissa}
Spiega come, per evitare dubbi di interpretazione nei punti più difficili o nelle partizioni più ingannevoli, sia stata sfruttata la documentazione sia a livello di analisi e progettazione, sia a livello di codice


%**************************************************************
\section{Priorità e criticità}

\subsection{Tecnologie software sfruttate}
Spiega quali sono le tecnologie che ho impiegato indicandone i punti di forza e utilità oltre a quelli critici
%**************************************************************
\subsection{Strumenti di collaborazione}
\subsubsection{Comunicazione}
Indica quali sono gli strumenti che sono in uso in Siav per la comunicazione e come sono stati sfruttati, indicandone anche pregi e difetti ed eventuali mancanze
\subsubsection{Documentazione}
Indica quali sono gli strumenti che sono in uso in Siav per la documentazione e come sono stati sfruttati, indicandone anche pregi e difetti ed eventuali mancanze
\subsubsection{Implementazione}
Indica quali sono gli strumenti che sono in uso in Siav per l'implementazione e come sono stati sfruttati, indicandone anche pregi e difetti ed eventuali mancanze
\subsubsection{Verifica}
Indica quali sono gli strumenti che sono in uso in Siav per la verifica e come sono stati sfruttati, indicandone anche pregi e difetti ed eventuali mancanze
%**************************************************************
\subsection{Esperienza utente}
Spiega come e perché da una proposta di "restyle" di un prototipo, io abbia proposto di creare una nuova partizione di prodotto ex novo concentrandomi sull'esperienza utente

%**************************************************************
\subsection{Disaccoppiamento}
Espone la situazione (drammatica) di alcune partizioni di software derivata dalla natura prototipale delle stesse o comunque da una ridotta progettazione e spiega il come io abbia voluto disaccoppiare la partizione del front-end dell'applicativo di mio interesse creando dapprima il mock-up di un servizio e poi andando a effettuare delle modifiche sul back-end che rispondessero alle specifiche REST in ambito CRUD.

%**************************************************************
\subsection{Versioni software}
Spiega l'importanza di verificare periodicamente la fattibilità di aggiornamenti LTS ed eventualmente effettuarli per non incorrere in problemi di aggiornamento a causa del gap che si crea in versioni di software "giovani"

\section{Prodotti dell'attività di stage}
%**************************************************************
\subsection{Analisi dei requisiti}
Espone la completezza del prodotto analisi dei requisiti rispetto agli obiettivi obbligatori e opzionali dello stage e quantifica la sua utilità

%**************************************************************
\subsection{Diario delle attività}
Espone come è stato redatto e quale sia la sua utilità

%**************************************************************
\subsection{Codice sorgente}
Espone cosa è stato prodotto a livello di sorgente