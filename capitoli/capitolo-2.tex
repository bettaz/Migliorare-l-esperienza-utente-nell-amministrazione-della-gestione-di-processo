% !TEX encoding = UTF-8
% !TEX TS-program = pdflatex
% !TEX root = ../tesi.tex

%**************************************************************
\chapter{Finalità del mio stage in Siav}
\label{cap:process-mining}
%**************************************************************
\section{Le possibilità offerte dallo stage}
%**************************************************************
\subsection{Libertà di conoscere}
Spiego perchè ho scelto questo stage e come tramite il lavoro su Bipod, gli stagisti vengono introdotti a nuove tecnologie e parallelamente ai concetti fondamentali del process mining comunque tenendo ampio grado di libertà di muoversi autonomamente.

%**************************************************************
\subsection{Nuovi concetti}
In questa sezione motivo la scelta dell'argomento di stage fra quelli proposti indicando anche la discussione specifica sull'obiettivo di stage con il responsabile e di come essa abbia poi influito sull'andamento.
%**************************************************************
\section{Principi di Process Mining}
%**************************************************************
\subsection{La gestione dei processi aziendali}
Spiega l'origine della necessità di gestire e monitorare i processi

%**************************************************************
\subsection{Da dato ad informazione}
Spiega l'importanza del passaggio da dato ad informazione introducendo il concetto di kpi e il perché sia altrettanto importante agevolarne quanto più possibile la creazione

%**************************************************************
\subsection{Kpi e dashboards nel process mining}
Introduce il concetto di dashboard, spiega perché dashboard e kpi aiutino l'attività di monitoraggio e verifica delle prestazioni 

%**************************************************************
\subsection{I ruoli nella gestione e amministrazione di processo}
Definisce i ruoli che sfruttano la piattaforma e l'utilità che percepiscono dal suo uso

%**************************************************************
\subsection{Le esigenze degli utenti}
Che servizio i vari ruoli si aspettano di ricevere
%**************************************************************
\section{Il ruolo di Bipod in Siav}
%**************************************************************
\subsection{Prerogative}
Visione a 360 gradi del software Bipod
%**************************************************************
\subsection{ProM - un valido concorrente}
Presenta gli aspetti ispiranti di un SW Open Source "concorrente" di Bipod
%**************************************************************
\subsection{La collaborazione con KaizenKey}
Spiega in che modo kaizen key collabora al progetto Bipod essendone il principale stakeholder esterno discutendo o ideando parte delle proposte prima che vengano inserite in un ciclo di sviluppo.
%**************************************************************
\subsection{Uno strumento flessibile}
Definisce quali sono le caratteristiche che Siav cerca in Bipod
%**************************************************************
\subsection{L'esperienza utente}
Definisce cosa abbiamo cerato di definire come obiettivo a livello di UX per rispondere alle esigenze enunciate in sezione 2.2.5