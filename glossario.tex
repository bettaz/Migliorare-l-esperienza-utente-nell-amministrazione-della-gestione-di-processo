
%**************************************************************
% Acronimi
%**************************************************************
\renewcommand{\acronymname}{Acronimi e abbreviazioni}

\newacronym[description={\glslink{RnDg}{Research and Development}}]
    {rnd}{RND}{Research and development}

\newacronym[description={\glslink{umlg}{Unified Modeling Language}}]
    {uml}{UML}{Unified Modeling Language}

\newacronym[description={\glslink{awsg}{Amazon Web Services}}]{aws}{AWS}{Amazon Web Services}

\newacronym[description={Machine Learning - Apprendimento automatico}]{ml}{ML}{Machine Learning}

\newacronym[description={\glslink{vpng}{Virtual Private Network}}]{vpn}{VPN}{Virtual Private Network}
\newacronym[description={\glslink{kpig}{Key Performance Indicator}}]{kpi}{KPI}{Key Performance Indicator}
\newacronym[description={\glslink{jvmg}{Java Virtual Machine}}]{jvm}{JVM}{Java Virtual Machine}
\newacronym[description={\glslink{uxg}{User Experience}}]{ux}{UX}{User Eperience}
\newacronym[description={\glslink{apig}{Application programming interface}}]{api}{API}{Application Programming Interface}
\newacronym[description={\glslink{ideg}{Integrated Development Environment}}]{ide}{IDE}{Integrated Development Environment}

%**************************************************************
% Glossario
%**************************************************************
\renewcommand{\glossaryname}{Glossario}

\newglossaryentry{RnDg}
{
    name=\glslink{rnd}{R\&D},
    text=Research and development
    sort=red,
    description={in informatica con il termine \emph{Application Programming Interface API} (ing. interfaccia di programmazione di un'applicazione) si indica ogni insieme di procedure disponibili al programmatore, di solito raggruppate a formare un set di strumenti specifici per l'espletamento di un determinato compito all'interno di un certo programma. La finalità è ottenere un'astrazione, di solito tra l'hardware e il programmatore o tra software a basso e quello ad alto livello semplificando così il lavoro di programmazione}
}

\newglossaryentry{umlg}
{
    name=\glslink{uml}{UML},
    text=UML,
    sort=uml,
    description={in ingegneria del software \emph{UML, Unified Modeling Language} (ing. linguaggio di modellazione unificato) è un linguaggio di modellazione e specifica basato sul paradigma object-oriented. L'\emph{UML} svolge un'importantissima funzione di ``lingua franca'' nella comunità della progettazione e programmazione a oggetti. Gran parte della letteratura di settore usa tale linguaggio per descrivere soluzioni analitiche e progettuali in modo sintetico e comprensibile a un vasto pubblico}
}

\newglossaryentry{scrum}
{
    name=SCRUM,
    text=SCRUM,
    sort=scrum,
    description={
    scrum è un framework di metodologie...
    }
}

\newglossaryentry{trello}{
    name=Trello,
    text=Trello,
    sort=trello,
    description={
        Trello è uno strumento di gesitione di progetto...
    }
}

\newglossaryentry{evernote}{
    name=Evernote,
    text=Evernote,
    sort=evernote,
    description={
        Evernote è uno strumento di scrittura collaborativa ...
    }
}

\newglossaryentry{sagemaker}{
    name=SageMaker,
    text=SageMaker,
    sort=sagemaker,
    description={
        SageMaker è un servizio \acrshort{aws}
    }
}

\newglossaryentry{awsg}{
    name=\acrshort{aws},
    text=AWS,
    sort=amazon web services,
    description={
        Gli Amazon Web Services sono ...
    }
}

\newglossaryentry{gitlab}{
    name=GitLab,
    text=GitLab,
    sort=gitlab,
    description={
        GitLab è un servizio di versionamento ...
    }
}
\newglossaryentry{vpng}{
    name=\acrshort{vpn},
    text=VPN,
    sort=virtual private network,
    description={
        Una VPN è una rete ...
    }
}

\newglossaryentry{kpig}{
    name=\acrshort{kpi},
    text=KPI,
    sort=key performance indicator,
    description={
        I Key Performance Indicator sono degli indici ...
    }
}

\newglossaryentry{jvmg}{
    name=\acrshort{jvm},
    text=JVM,
    sort=java virtual machine,
    description={
        La java virtual machine è ..
    }
}

\newglossaryentry{uxg}{
    name=\acrshort{ux},
    text=UX,
    sort=user experience,
    description={
        La User experience ...
    }
}
\newglossaryentry{apig}{
    name=\acrshort{api},
    text=API,
    sort=application programming interface,
    description={
        Per application programming interface si intende ...
    }
}
\newglossaryentry{fork}{
    name=fork,
    text=fork,
    sort=fork,
    description={
        Nell'ambito della gestione delle versioni, una fork è un operazione....
    }
}
\newglossaryentry{ideg}{
    name=IDE,
    text=IDE,
    sort=integrated development environment,
    description={
        Un ambiente di sviluppo integrato è ...
    }
}