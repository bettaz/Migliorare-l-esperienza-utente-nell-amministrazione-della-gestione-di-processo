
%**************************************************************
% Acronimi
%**************************************************************
\renewcommand{\acronymname}{Acronimi e abbreviazioni}

\newacronym[description={\glslink{RnDg}{Research and Development}}]
    {rnd}{RND}{Research and development}

\newacronym[description={\glslink{umlg}{Unified Modeling Language}}]
    {uml}{UML}{Unified Modeling Language}

\newacronym[description={\glslink{awsg}{Amazon Web Services}}]{aws}{AWS}{Amazon Web Services}

\newacronym[description={Machine Learning - Apprendimento automatico}]{ml}{ML}{Machine Learning}

\newacronym[description={\glslink{vpng}{Virtual Private Network}}]{vpn}{VPN}{Virtual Private Network}
\newacronym[description={\glslink{kpig}{Key Performance Indicator}}]{kpi}{KPI}{Key Performance Indicator}
\newacronym[description={\glslink{jvmg}{Java Virtual Machine}}]{jvm}{JVM}{Java Virtual Machine}
\newacronym[description={\glslink{uxg}{User Experience}}]{ux}{UX}{User Eperience}
\newacronym[description={\glslink{apig}{Application programming interface}}]{api}{API}{Application Programming Interface}
\newacronym[description={\glslink{ideg}{Integrated Development Environment}}]{ide}{IDE}{Integrated Development Environment}
\newacronym[description={\glslink{pocg}{Proof Of Concept}}]{poc}{ POC}{Proof Of Concept}
\newacronym[description={\glslink{cig}{Continuous Integration}}]{ci}{CI}{Continuous Integration}
\newacronym[description={\glslink{restg}{REpresentational State Transfer}}]{rest}{REST}{REpresentational State Transfer}
\newacronym[description={\glslink{xmlg}{eXteded Mark-up Language}}]{xml}{XML}{eXtended Mark-up Language}
\newacronym[description={K Desktop Environment}]{kde}{KDE}{K Desktop Environment}
\newacronym[description={\glslink{xliffg}{\acrshort{xml} Localization Interchange File Format}}]{xliff}{XLIFF}{\acrshort{xml} Localization Interchange File Format}
\newacronym[description={\glslink{jsong}{JavaScript Object Notation}}]{json}{JSON}{Javascript Object Notation}
\newacronym[description={\glslink{httpg}{HyperText Transfer Protocol}}]{http}{HTTP}{HyperText Transfer Protocol}
\newacronym[description={\glslink{dbmsg}{DataBase Management System}}]{dbms}{DBMS}{Database Management System}
%**************************************************************
% Glossario
%**************************************************************
\renewcommand{\glossaryname}{Glossario}

\newglossaryentry{RnDg}
{
    name=\glslink{rnd}{R\&D},
    text=Research and development
    sort=red,
    description={
        RnD ..
    }
}

\newglossaryentry{umlg}
{
    name=\glslink{uml}{UML},
    text=UML,
    sort=uml,
    description={in ingegneria del software \emph{UML, Unified Modeling Language} (ing. linguaggio di modellazione unificato) è un linguaggio di modellazione e specifica basato sul paradigma object-oriented. L'\emph{UML} svolge un'importantissima funzione di ``lingua franca'' nella comunità della progettazione e programmazione a oggetti. Gran parte della letteratura di settore usa tale linguaggio per descrivere soluzioni analitiche e progettuali in modo sintetico e comprensibile a un vasto pubblico}
}

\newglossaryentry{scrum}
{
    name=SCRUM,
    text=SCRUM,
    sort=scrum,
    description={
    scrum è un framework di metodologie...
    }
}

\newglossaryentry{trello}{
    name=Trello,
    text=Trello,
    sort=trello,
    description={
        Trello è uno strumento di gesitione di progetto...
    }
}

\newglossaryentry{evernote}{
    name=Evernote,
    text=Evernote,
    sort=evernote,
    description={
        Evernote è uno strumento di scrittura collaborativa ...
    }
}

\newglossaryentry{sagemaker}{
    name=SageMaker,
    text=SageMaker,
    sort=sagemaker,
    description={
        SageMaker è un servizio \acrshort{aws}
    }
}

\newglossaryentry{awsg}{
    name=\acrshort{aws},
    text=AWS,
    sort=amazon web services,
    description={
        Gli Amazon Web Services sono ...
    }
}

\newglossaryentry{gitlab}{
    name=GitLab,
    text=GitLab,
    sort=gitlab,
    description={
        GitLab è un servizio di versionamento ...
    }
}
\newglossaryentry{vpng}{
    name=\acrshort{vpn},
    text=VPN,
    sort=virtual private network,
    description={
        Una VPN è una rete ...
    }
}

\newglossaryentry{kpig}{
    name=\acrshort{kpi},
    text=KPI,
    sort=key performance indicator,
    description={
        I Key Performance Indicator sono degli indici ...
    }
}

\newglossaryentry{jvmg}{
    name=\acrshort{jvm},
    text=JVM,
    sort=java virtual machine,
    description={
        La java virtual machine è ..
    }
}

\newglossaryentry{uxg}{
    name=\acrshort{ux},
    text=UX,
    sort=user experience,
    description={
        La User experience ...
    }
}
\newglossaryentry{apig}{
    name=\acrshort{api},
    text=API,
    sort=application programming interface,
    description={
        in informatica con il termine \emph{Application Programming Interface API} (ing. interfaccia di programmazione di un'applicazione) si indica ogni insieme di procedure disponibili al programmatore, di solito raggruppate a formare un set di strumenti specifici per l'espletamento di un determinato compito all'interno di un certo programma. La finalità è ottenere un'astrazione, di solito tra l'hardware e il programmatore o tra software a basso e quello ad alto livello semplificando così il lavoro di programmazione
        }
}
\newglossaryentry{fork}{
    name=Fork,
    text=fork,
    sort=fork,
    description={
        Nell'ambito della gestione delle versioni, una fork è un operazione....
    }
}
\newglossaryentry{ideg}{
    name=\acrshort{ide},
    text=IDE,
    sort=integrated development environment,
    description={
        Un ambiente di sviluppo integrato è ...
    }
}
\newglossaryentry{pocg}{
    name=\acrshort{poc},
    text=POC,
    sort=proof of concept,
    description={
        Un proof of concept è...
    }
}
\newglossaryentry{jsdoc}{
    name=JSDoc,
    text=JSDoc,
    sort=javascript docstring,
    description={
        Con JSDoc si intende...
    }
}
\newglossaryentry{markdown}{
    name=Markdown,
    text=Markdown,
    sort=markdown,
    description={
        Il linguaggio markdown ...
    }
}
\newglossaryentry{diff}{
    name=Diff,
    text=diff,
    sort=diff,
    description={
        \'E un comando git ...
    }
}
\newglossaryentry{cig}{
    name=\acrshort{ci},
    text=CI,
    sort=continuous integration,
    description={
        La continunous integration è una tecnica ...
    }
}
\newglossaryentry{restg}{
    name=\acrshort{rest},
    text=Representational State Transfer,
    sort=representational state transfer,
    description={
        I principi REST, sono ..
    }
}
\newglossaryentry{xmlg}{
    name=\acrshort{xml},
    text=eXtended Mark-up Language,
    sort=extended markup language,
    description={
        XML è un linguaggio ...
    }
}
\newglossaryentry{discover}{
    name=Discover,
    text=Discover,
    sort=discover,
    description=Discover è il software \textit{manager} di \acrshort{kde}
}
\newglossaryentry{xliffg}{
    name=\acrshort{xliff},
    text=XLIFF,
    sort=xml localization interchange file format,
    description={
        XLIFF è ...
    }
}
\newglossaryentry{office}{
    name=Office 365,
    text=Office 365,
    sort=office 365,
    description={
        Office 365 è ...
    }
}
\newglossaryentry{gradle}{
    name=Gradle,
    text=Gradle,
    sort=gradle,
    description={
        Gradle...
    }
}
\newglossaryentry{solid}{
    name=SOLID Principles,
    text=SOLID,
    sort=solid,
    description={
        I principi SOLID ...
    }
}
\newglossaryentry{jsong}{
    name=JSON,
    text=JSON,
    sort=json,
    description={
        JSON è...
    }
}
\newglossaryentry{httpg}{
    name=HTTP,
    text=HTTP,
    sort=hypertext transfer protocol,
    description={
        HTTP è
    }
}
\newglossaryentry{dbmsg}{
    name=DBMS,
    text=DBMS,
    sort=database management system,
    description={
        I DBMS sono ...
    }
}
\newglossaryentry{commit}{
    name=commit,
    text=commit,
    sort=commit,
    description={
        Un commit è...
    }
}