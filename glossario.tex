
%**************************************************************
% Acronimi
%**************************************************************
\renewcommand{\acronymname}{Acronimi e abbreviazioni}

\newacronym[description={\glslink{RnDg}{Research and Development}}]
    {rnd}{RND}{Research and development}

\newacronym[description={\glslink{awsg}{Amazon Web Services}}]{aws}{AWS}{Amazon Web Services}

\newacronym[description={Machine Learning - Apprendimento automatico}]{ml}{ML}{Machine Learning}

\newacronym[description={\glslink{vpng}{Virtual Private Network}}]{vpn}{VPN}{Virtual Private Network}
\newacronym[description={\glslink{kpig}{Key Performance Indicator}}]{kpi}{KPI}{Key Performance Indicator}
\newacronym[description={\glslink{jvmg}{Java Virtual Machine}}]{jvm}{JVM}{Java Virtual Machine}
\newacronym[description={\glslink{uxg}{User Experience}}]{ux}{UX}{User Eperience}
\newacronym[description={\glslink{apig}{Application programming interface}}]{api}{API}{Application Programming Interface}
\newacronym[description={\glslink{ideg}{Integrated Development Environment}}]{ide}{IDE}{Integrated Development Environment}
\newacronym[description={\glslink{pocg}{Proof Of Concept}}]{poc}{ POC}{Proof Of Concept}
\newacronym[description={\glslink{cig}{Continuous Integration}}]{ci}{CI}{Continuous Integration}
\newacronym[description={\glslink{restg}{REpresentational State Transfer}}]{rest}{REST}{REpresentational State Transfer}
\newacronym[description={\glslink{xmlg}{eXteded Mark-up Language}}]{xml}{XML}{eXtended Mark-up Language}
\newacronym[description={K Desktop Environment}]{kde}{KDE}{K Desktop Environment}
\newacronym[description={\glslink{xliffg}{\acrshort{xml} Localization Interchange File Format}}]{xliff}{XLIFF}{\acrshort{xml} Localization Interchange File Format}
\newacronym[description={\glslink{jsong}{JavaScript Object Notation}}]{json}{JSON}{Javascript Object Notation}
\newacronym[description={\glslink{httpg}{HyperText Transfer Protocol}}]{http}{HTTP}{HyperText Transfer Protocol}
\newacronym[description={\glslink{dbmsg}{DataBase Management System}}]{dbms}{DBMS}{Database Management System}
%**************************************************************
% Glossario
%**************************************************************
\renewcommand{\glossaryname}{Glossario}

\newglossaryentry{RnDg}
{
    name=\glslink{rnd}{R\&D},
    text=Research and development
    sort=red,
    description={
        Per \acrfull{rnd} si intende il reparto "Ricerca e Sviluppo" ovvero quello che si occupa di effettuare ricerca per portare innovazione nei prodotti o nei processi aziendali
    }
}

\newglossaryentry{scrum}
{
    name=Scrum,
    text=Scrum,
    sort=scrum,
    description={
        Scrum è un framework di metodologie \textit{Agile} nato per fornire un metodo di organizzazione dei progetti\footcite{sommerville:IngegneriaDelSoftware}
    }
}

\newglossaryentry{trello}{
    name=Trello,
    text=Trello,
    sort=trello,
    description={
        Trello è uno strumento visivo che permette di gestire flessibilmente i progetti\footcite{site:trello}
    }
}

\newglossaryentry{evernote}{
    name=Evernote,
    text=Evernote,
    sort=evernote,
    description={
        Evernote è uno strumento di scrittura collaborativa basato sull'uso di taccuini contenenti note di ogni genere
    }
}

\newglossaryentry{sagemaker}{
    name=SageMaker,
    text=SageMaker,
    sort=sagemaker,
    description={
        SageMaker è un servizio \acrshort{aws} che permette la gestione completa del \textit{lifecycle} di modelli di \acrlong{ml}
    }
}

\newglossaryentry{awsg}{
    name=\acrshort{aws},
    text=AWS,
    sort=amazon web services,
    description={
        Amazon Web Services è un'azienda di proprietà del gruppo Amazon che fornisce servizi di \textit{cloud computing}
    }
}

\newglossaryentry{gitlab}{
    name=GitLab,
    text=GitLab,
    sort=gitlab,
    description={
        GitLab è un servizio di versionamento Git che comprente anche un \textit{tool} di ticketing al suo interno
    }
}
\newglossaryentry{vpng}{
    name=\acrshort{vpn},
    text=VPN,
    sort=virtual private network,
    description={
        Una VPN è una rete di comunicazione privata che sfrutta un protocollo di trasmissione pubblico per effettuare la comunicazione. Viene spesso impiegata per accedere alla \textit{intranet} aziendale da posizione remote
    }
}

\newglossaryentry{kpig}{
    name=\acrshort{kpi},
    text=KPI,
    sort=key performance indicator,
    description={
        I Key Performance Indicator sono degli indici di prestazione. Essi servono a evidenziare l'andamento di una particolare caratteristica di un processo aziendale
    }
}

\newglossaryentry{jvmg}{
    name=\acrshort{jvm},
    text=JVM,
    sort=java virtual machine,
    description={
        La java virtual machine è una parte della \textit{runtime} Java che si occupa di eseguire i programmi compilati come bytecode in pacchetti
    }
}

\newglossaryentry{uxg}{
    name=\acrshort{ux},
    text=UX,
    sort=user experience,
    description={
        La User Experience rappresenta tutte le interazioni e rapporti che un utente ha con un prodotto
    }
}
\newglossaryentry{apig}{
    name=\acrshort{api},
    text=API,
    sort=application programming interface,
    description={
        in informatica con il termine \emph{Application Programming Interface API} (ing. interfaccia di programmazione di un'applicazione) si indica ogni insieme di procedure disponibili al programmatore, di solito raggruppate a formare un set di strumenti specifici per l'espletamento di un determinato compito all'interno di un certo programma. La finalità è ottenere un'astrazione, di solito tra l'hardware e il programmatore o tra software a basso e quello ad alto livello semplificando così il lavoro di programmazione
        }
}
\newglossaryentry{fork}{
    name=Fork,
    text=fork,
    sort=fork,
    description={
        Nell'ambito della gestione delle versioni, una \textit{fork} è un operazione di biforcazione che porta ad originare un nuovo \textit{repository} a partire da una specifica versione di un software
    }
}
\newglossaryentry{ideg}{
    name=\acrshort{ide},
    text=IDE,
    sort=integrated development environment,
    description={
        Un ambiente di sviluppo integrato è un software che facilita e potenzia la modifica di \textit{file} di codice sorgente tramite strumenti e componenti aggiuntive configurabili
    }
}
\newglossaryentry{pocg}{
    name=\acrshort{poc},
    text=POC,
    sort=proof of concept,
    description={
        Un proof of concept è uno strumento dimostrativo della fattibilità di una funzionalità che deve ancora essere sviluppata
    }
}
\newglossaryentry{jsdoc}{
    name=JSDoc,
    text=JSDoc,
    sort=javascript docstring,
    description={
        \'E uno standard per la stesura di documentazione di codice sorgente JavaScript che facilita la successiva creazione automatica di documentazione di riferimento
    }
}
\newglossaryentry{markdown}{
    name=Markdown,
    text=Markdown,
    sort=markdown,
    description={
        Il linguaggio markdown è un linguaggio di formattazione dei testi che fa della semplicità la sua forza
    }
}
\newglossaryentry{diff}{
    name=Diff,
    text=diff,
    sort=diff,
    description={
        \'E un comando git che permette di controllare quali righe di un file sono cambiate fra una versione e un'altra
    }
}
\newglossaryentry{cig}{
    name=\acrshort{ci},
    text=CI,
    sort=continuous integration,
    description={
        La continunous integration è una tecnica di sviluppo della disciplina \textit{extreme programming}. Prevede l'inoltro frequente delle modifiche software effettuate sul \textit{repository} condiviso qualora esse siano valide
    }
}
\newglossaryentry{restg}{
    name=\acrshort{rest},
    text=Representational State Transfer,
    sort=representational state transfer,
    description={
        REpresentational State Transfer è uno stile architetturale atto ad eliminare lo stato nelle comunicazioni HTTP
    }
}
\newglossaryentry{xmlg}{
    name=\acrshort{xml},
    text=eXtended Mark-up Language,
    sort=extended markup language,
    description={
        XML è un metalinguaggio di definizone di modelli di markup
    }
}
\newglossaryentry{discover}{
    name=Discover,
    text=Discover,
    sort=discover,
    description=Discover è il software \textit{manager} di \acrshort{kde}
}
\newglossaryentry{xliffg}{
    name=\acrshort{xliff},
    text=XLIFF,
    sort=xml localization interchange file format,
    description={
        XLIFF è un formato \acrshort{xml} per la definizione di traduzioni. Contiene una località di origine e una di destinazione. Inoltre contiene una lista di termini nella lingua della località di origine con la corrispettiva traduzione
    }
}
\newglossaryentry{office}{
    name=Office 365,
    text=Office 365,
    sort=office 365,
    description={
        Office 365 è la suite di tools di Office Automation in \textit{cloud} sviluppata da Microsoft
    }
}
\newglossaryentry{gradle}{
    name=Gradle,
    text=Gradle,
    sort=gradle,
    description={
        Gradle è un tool di \textit{build automation}
    }
}
\newglossaryentry{solid}{
    name=SOLID Principles,
    text=SOLID,
    sort=solid,
    description={
        L'acronimo rappresenta i principi descritti da Robert Martin per la programmazione ad oggetti: singola responsabilità, apertura all'estensione e chiusura alle modifiche, principio di sostituzione di Liskov, segregazione delle interfacce, inversione delle dipendenze
    }
}
\newglossaryentry{jsong}{
    name=JSON,
    text=JSON,
    sort=json,
    description={
        \'E un formato dati di tipo dizionario utile alla rappresentazione su \textit{filesystem} di oggetti JavaScript
    }
}
\newglossaryentry{httpg}{
    name=HTTP,
    text=HTTP,
    sort=hypertext transfer protocol,
    description={
        HTTP è un protocollo di comunicazione ideato per il trasferimento di ipertesti
    }
}
\newglossaryentry{dbmsg}{
    name=DBMS,
    text=DBMS,
    sort=database management system,
    description={
        \'E un software che permette di gestire una basi di dati che isola lo stato fisico della memorizzazione fornendo dei metodi per accedere alla stessa.
    }
}
\newglossaryentry{commit}{
    name=commit,
    text=commit,
    sort=commit,
    description={
        Un commit è la registrazione di una versione software in un \textit{repository}. Identificato da un codice esadecimale univoco, può essere accompagnato da un messaggio che identifichi le novità che quella versione porta all'archivio
    }
}