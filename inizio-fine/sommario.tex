% !TEX encoding = UTF-8
% !TEX TS-program = pdflatex
% !TEX root = ../tesi.tex

%**************************************************************
% Sommario
%**************************************************************
\cleardoublepage
\phantomsection
\pdfbookmark{Sommario}{Sommario}
\begingroup
\let\clearpage\relax
\let\cleardoublepage\relax
\let\cleardoublepage\relax

%**************************************************************
\chapter*{Presentazione dell'elaborato}
\begin{description}
    \item[{\hyperref[cap:introduzione]{Il primo capitolo}}] presenta il contesto aziendale nel quale ho svolto l'attività di stage
    \item[{\hyperref[cap:process-mining]{Il secondo capitolo}}] descrive da dove nasce l'esigenza di monitorare e gestire processi tramite l'impiego di software
    
    \item[{\hyperref[cap:modalita-svolgimento]{Il terzo capitolo}}] approfondisce le metodologie e il prodotto software oggetto di stage
    
    \item[{\hyperref[cap:performance-stage]{Il quarto capitolo}}] mostra come l'esperienza di stage ha influito sullo studente
    
\end{description}

Riguardo la stesura del testo, relativamente al documento sono state adottate le seguenti convenzioni tipografiche:
\begin{itemize}
	\item gli acronimi, le abbreviazioni e i termini ambigui o di uso non comune menzionati vengono definiti nel glossario, situato alla fine del presente documento;
	\item per la prima occorrenza dei termini riportati nel glossario viene utilizzata la seguente nomenclatura: \emph{parola}\glsfirstoccur;
	\item i termini in lingua straniera o facenti parti del gergo tecnico sono evidenziati con il carattere \emph{corsivo}.
\end{itemize}


%\vfill
%
%\selectlanguage{english}
%\pdfbookmark{Abstract}{Abstract}
%\chapter*{Abstract}
%
%\selectlanguage{italian}

\endgroup			

\vfill

